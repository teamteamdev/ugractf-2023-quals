\documentclass[a4paper]{article}
\usepackage{amsmath,geometry,unicode-math}
\usepackage[shortlabels]{enumitem}

\newcommand{\const}{\mathrm{const}}
\newcommand{\abs}[1]{\left\lvert#1\right\rvert}
\newcommand{\floor}[1]{\left\lfloor#1\right\rfloor}
\newcommand{\ceil}[1]{\left\lceil#1\right\rceil}
\newcommand{\dd}{\mathrm{d}}

\makeatletter
\newenvironment{sqcases}{%
	\matrix@check\sqcases\env@sqcases
}{%
	\endarray\right.%
}
\def\env@sqcases{%
	\let\@ifnextchar\new@ifnextchar
	\left\lbrack
	\def\arraystretch{1.2}%
	\array{@{}l@{\quad}l@{}}%
}
\makeatother

\setlength{\parindent}{0pt}
\setlength{\parskip}{5pt}
\setmainfont{CMU Serif}
\widowpenalties 1 10000
\raggedbottom

\title{Олимпиада им. У. Ц. Уцуги \\ \Large 9--11 класс}
\date{30 февраля 2020 г.}

\begin{document}
	\maketitle

	\textbf{Задача 1.} Рассмотрим обобщенную игру для двух игроков в крестики-нолики на поле $n \times n \times \dots \times n = n^w$: первый и второй игрок по очереди ставят $X$ и $O$ соответственно в клетки изначально пустой $w$-мерной таблицы. Игрок выигрывает, если после его хода найдется упорядоченный набор из $n$ клеток с одинаковыми символами, такой, что по каждой координате у всех клеток набора эта координата либо совпадает, либо монотонно убывает или возрастает. Если таблица оказывается заполнена целиком, но ни один игрок не выигрыл, объявляется ничья.

	Считая, что оба игрока играют оптимально:

	\begin{enumerate}[a)]
		\item (1 балл) Покажите, что при $w = 2, n \le 2$ выигрывает первый игрок.
		\item (2 балла) Покажите, что при $w = 2, n \ge 6$ второй игрок может достичь ничьи независимо от ходов первого.
		\item (по 0.5 балла, итого 1.5 баллов) Пусть $w = 2$. Покажите, что при $n = 3, 4, 5$ второй игрок может достичь ничьи независимо от ходов первого.
		\item (3 балла) Пусть $w$ произвольно. Покажите, что начиная с некоторого достаточно большого $n$ первый игрок всегда выигрывает.
		\item (2 балла) Пусть $w$ произвольно. Покажите, что если при $n$ первый игрок всегда выигрывает, то он выигрывает и при $n+1$.
	\end{enumerate}


	\textbf{Задача 2.} Выясните, имеет ли следующее уравнение решение в целых числах при данных условиях:

	\begin{equation*}
		\frac{x}{y + z} + \frac{y}{x + z} + \frac{z}{x + y} = n
	\end{equation*}

	\begin{enumerate}[a)]
		\item (1 балл) $n = 0$
		\item (1 балл) $n = 1$
		\item (1 балл) $n = 2$
		\item (2 балла) $n$ -- положительное нечетное число.
		\item (3 балла) $n$ -- положительное четное число.
	\end{enumerate}


	\textbf{Задача 3.} Определим бесконечную последовательность целых чисел следующим способом: $a_1 = k$, а далее $a_n = f(m a_{n-1} + 1)$, где $f(x)$ -- максимальный нечетный делитель числа $x$. Здесь $k, m$ -- параметры.

	Обозначим за $b_k$ минимум последовательности $\{a_n\}$ в зависимости от $k$. Посчитайте максимум $b_k$ по всем $k$, если:

	\begin{enumerate}[a)]
		\item (1 балл) $m = 10$.
		\item (1 балл) $m = 5$.
		\item (3 балла) $m$ -- натуральное, $m \ne 3$.
		\item (4 балла) $m = 3$.
	\end{enumerate}


	\textbf{Задача 4.} Определим последовательность действительных функций действительного переменного так: пусть $f_0(x) = \mathrm{1}_{-1 < x < 1}$, а далее

	\begin{equation*}
		f_n(x) = \frac{1}{1+2n} \int_{x - \frac{1}{2+4n}}^{x + \frac{1}{2+4n}} f_{n-1}(x) \dd x.
	\end{equation*}

	\begin{enumerate}[a)]
		\item (2 балла) Покажите, что при всех $n$: $f_n(0) = 1$.
		\item (3 балла) Покажите, что при всех $n$:
		\begin{equation*}
			\int_{-\infty}^{+\infty} \frac{\sin x}{x} \cdot \frac{\sin (x/3)}{x/3} \cdot \dots \cdot \frac{\sin \frac{x}{2n+1}}{x/(2n+1)} \dd x = \pi.
		\end{equation*}
	\end{enumerate}


	\textbf{Задача 5.} (4 балла) Решите систему функциональных уравнений относительно гладких действительных функций четырех переменных в общем виде, считая функции $\rho, J_x, J_y, J_z$ параметрами:

	\begin{gather*}
		\begin{cases}
			\dfrac{\partial D_x}{\partial x}(t, x, y, z) + \dfrac{\partial D_y}{\partial y}(t, x, y, z) + \dfrac{\partial D_z}{\partial z}(t, x, y, z) = \rho(t, x, y, z) \\
			\dfrac{\partial B_x}{\partial x}(t, x, y, z) + \dfrac{\partial B_y}{\partial y}(t, x, y, z) + \dfrac{\partial B_z}{\partial z}(t, x, y, z) = 0 \\
			\dfrac{\partial E_y}{\partial z}(t, x, y, z) - \dfrac{\partial E_z}{\partial y}(t, x, y, z) = \dfrac{\partial B_x}{\partial t}(t, x, y, z) \\
			\dfrac{\partial E_z}{\partial x}(t, x, y, z) - \dfrac{\partial E_x}{\partial z}(t, x, y, z) = \dfrac{\partial B_y}{\partial t}(t, x, y, z) \\
			\dfrac{\partial E_x}{\partial y}(t, x, y, z) - \dfrac{\partial E_y}{\partial x}(t, x, y, z) = \dfrac{\partial B_z}{\partial t}(t, x, y, z) \\
			\dfrac{\partial E_z}{\partial y}(t, x, y, z) - \dfrac{\partial E_y}{\partial z}(t, x, y, z) = J_x(t, x, y, z) + \dfrac{\partial D_x}{\partial t}(t, x, y, z) \\
			\dfrac{\partial E_x}{\partial z}(t, x, y, z) - \dfrac{\partial E_z}{\partial x}(t, x, y, z) = J_y(t, x, y, z) + \dfrac{\partial D_y}{\partial t}(t, x, y, z) \\
			\dfrac{\partial E_y}{\partial x}(t, x, y, z) - \dfrac{\partial E_x}{\partial y}(t, x, y, z) = J_z(t, x, y, z) + \dfrac{\partial D_z}{\partial t}(t, x, y, z)
		\end{cases}
	\end{gather*}

	\begin{center}
		Желаем удачи!
	\end{center}
\end{document}
